\documentclass[tikz,border=5pt]{standalone}
% % % % % % % % % % % % % % % % % % % % % % % % % %
%    FIGURA C16 DEL APENDICE DE DIAGRAMA DE BLOQUES
%
% Luego separar las figuras en archivos usando el comando:
%
% pdftk figc16.pdf burst
%
% % % % % % % % % % % % % % % % % % % % % % % % % %


\usepackage[utf8]{inputenc}
\usepackage{ucs}
\usepackage[spanish,activeacute]{babel}
\usetikzlibrary{positioning}

% % % % % % % % % % % % % % % % % % %
%ACTIVAR CUANDO SE NECESITE HACER DB
\usepackage{schemabloc}
\usepackage{blox}
\usepackage{tcolorbox}
% % % % % % % % % % % % % % % % % % %


\begin{document}
\begin{tikzpicture}
	\bXInput{entradaR}
	\bXBloc{bloqueGt}{$K_m$}{entradaR}
	\bXLink{entradaR}{bloqueGt} \bXLinkName[0.6]{entradaR}{$R(s)$}
	\bXCompSum*[5]{compara}{bloqueGt}{}{-}{}{}
	\bXLink[$R_t(s)$]{bloqueGt}{compara}
	%cadena GH
	\bXBloc[3]{bloqueC}{$C(s)$}{compara}
	\bXLink[$E_a(s)$]{compara}{bloqueC}
	\bXBloc[3]{bloqueGv}{$G_v(s)$}{bloqueC}
	\bXLink[$Y_r(s)$]{bloqueC}{bloqueGv}
	\bXBloc[3]{bloqueGp}{$G_p(s)$}{bloqueGv}
	\bXLink[$U(s)$]{bloqueGv}{bloqueGp}
	%entrada L
	\bXCompSum*{sumaL}{bloqueGp}{}{}{}{}
	\bXLink{bloqueGp}{sumaL}
	\bXInput{entradaL}
	\bXBranchy[-4]{bloqueGv}{entra_l} %arriba y al medio de Gv-Gp crea un nodo con nombre entra_l
	\bXBloc[4.5]{bloqueGl}{$G_l(s)$}{entra_l}
	\bXLink[$L(s)$]{entra_l}{bloqueGl}
	\bXLinkxy{bloqueGl}{sumaL}
	\bXOutput[2.5]{salida_y}{sumaL}
	\bXLink{sumaL}{salida_y} \bXLinkName[0.6]{salida_y}{$Y(s)$}
	%Cadena de Retorno
	\bXBranchy[5]{sumaL-salida_y}{retorna_y}
	\bXCompSum*[0]{sumaRuido}{retorna_y}{}{}{}{}
	\bXBranchy[0.35]{sumaL-salida_y}{sale_y}
	\bXLinkxy{sale_y}{sumaRuido}
	\bXBranchx[3]{sumaRuido}{N}
	\bXLink{N}{sumaRuido} \bXLinkName[0.6]{N}{$N(s)$}
	\bXBloc[-15]{bloqueGm}{$Gm(s)$}{sumaRuido}
	\bXLink{sumaRuido}{bloqueGm}
	\bXLinkxy[$Y_m(s)$]{bloqueGm}{compara}
	%Armo un rectángulo y escribo la palabra Planta
	\draw [dashed] (11.4,-1)--(11.4,2.3)--(8.45,2.3)--(8.45,-1)--(11.4,-1);
	\draw (10.9,2.3) node[align=left, above]{Planta};
\end{tikzpicture}
\end{document}
